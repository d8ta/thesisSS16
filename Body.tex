\section{Introduction}
Today the need for stable webapplications is vital for companys and startups. Established businesses need to be
available for their customers and users all the time. For Startups the availability of their services can be the difference
between failure and success. Webapplications and services needs to change and scale over time. Sometimes it is the need
to open up your service for more customers, like Runtastic, a small company from Linz/austria. They developed
a fitness application whose userbase was growing in a short periode of time. Somtimes the customerbase demands a 24/7
availability, for example the sales platform Amazon or the social network Facebook. When Facebook took an outage on September 28 2015,
this incident got a huge ammount of interest from local, national and international media (\url{http://www.bbc.com/news/world-us-canada-34383655}.

Downtime is costly. In the best case it only costs money, in the worst case
it can be the failure of your business or startup. All the mentioned services and events have one thing in common. To provide this kind of zero
downtime and continuous change in your webapplication you need a reliable and stable automatisation for your services. Development operations are
right now one of the major topics in webdevelopment. If used properly, DevOps can provide this kind of zero downtime and progress in an application
(\cite{humble2010continuous} \cite{duvall2007continuous}). To do so DevOps can use a variaty of techniques and methods to continuously
integrate, automatic test and deploy an applications (\cite{meyer2014continuous}).
Continuous Integration and development is based on the method of agile software development and extrem programming
(\cite{lindstrom2004extreme}). There is a reliable base of knowledge and literature for Continous Integration today
(\cite{schaefer2013continuous}), (\cite{fowler2006continuous}) (\cite{fowler2012continuous}).

However, some of the newer methods, for example MEAN stack or NodeJS are not yet fully covered.
The Question i will research in this thesis is, how can a modern webapplication automatized and deployed to continously integrate features and
changes. It is my goal to show how to establish a modern development operation cycle for webapplications using Node.js and the MEAN stack.
I will citeing the standard literature and methods for continuous integration and deployment and show how to build them in a safe and stable
way with Node.js and other Parts of the MEAN stack.
For this i will study the literatur that is available for continuous integration and autmatization to build a similar system with NodeJs.
To do so i will describe how to automatize builds and tests and how to use containerization to simplify the process. Based on the
implementation cycle of the neolexon webapplication, i will provide data to prove that this setup is capable of continous integration and deployment
ant that it runs in a stable and reliable.

\newpage

\section{Introduction to Agile Development and the Role of Development Operations in Modern Web Applications
}
\label{section:Introduction to Agile Development and the Role of Development Operations in Modern Web Applications
}
\newpage

\section{The Foundations of Continuous Integration and Automatization in Development Operations
}
\label{section:The Foundations of Continuous Integration and Automatization in Development Operations
}
\newpage

\section{Development and Deployment with MEAN
}
\label{section:Development and Deployment with MEAN
}
\newpage

\section{Test Driven Development with MEAN
}
\label{section:Test Driven Development with MEAN
}
\newpage

\section{Containerization and Deployment with Docker
}
\label{section:Containerization and Deployment with Docker
}
\newpage

\section{Building Continuous Integrations and Deployment with NodeJS
}
\label{section:Building Continuous Integrations and Deployment with NodeJS
}
\newpage


% \begin{figure}[h!]
%   \centering
%       \includegraphics[width=0.4\textwidth]{images/Perlin-Coherent.png}
%   \caption{Just some example figure}
% \end{figure}



% \subparagraph{subparagraph}
% \footcite{meyer2014continuous}
%
% \begin{itemize}
%   \item Itemlist 1
%   \item Itemlist 2
% \end{itemize} \cite{cranorplatform}
%
% \section{Next Section}
% \label{section:Label}
%
% \textit{Texit Option}
%
% \begin{figure}[h!]
%   \centering
%       \includegraphics[width=0.2\textwidth]{images/Julia-Fractal.png}
%   \caption{Exampelimage}
% \end{figure}
%
% \subparagraph{Unforgeability}
% \label{subp:subparagraph_name}
%
% Graphic by \url{http://en.wikipedia.org/wiki/Pretty_Good_Privacy#/media/File:PGP_diagram.svg
