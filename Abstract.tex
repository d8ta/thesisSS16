\subsection*{Abstract}

Today the need for stable webapplications is vital for companys and startups. Established businesses need to be
available for their customers and users all the time. For Startups the availability of their services can be the difference
between failure and success. Webapplications and services needs to change and scale over time. Sometimes it is the need
to open up your service for more customers, like Runtastic, a small company from Linz/austria. They developed
a fitness application whose userbase was growing in a short periode of time. Somtimes the customerbase demands a 24/7
availability, for example the sales platform Amazon.

Downtime is costly. In the best case it only costs money, in the worst case
it can be the failure of your business or startup. All the mentioned services and events have one thing in common. To provide this kind of zero downtime and continuous change in
your webapplication you need a reliable and stable automatisation for your services. Development operations are right now one of
the major topics in webdevelopment. If used properly, DevOps can provide this kind of zero downtime and progress in an application
(\cite{humble2010continuous} \cite{duvall2007continuous}). To do so DevOps can use a variaty of techniques and methods to continuously
integrate, automatic test and deploy an applications (\cite{meyer2014continuous}).
Continuous Integration and development is based on the method of agile software development and extrem programming
(\cite{lindstrom2004extreme}). There is a reliable base of knowledge and literature for Continous Integration
(\cite{schaefer2013continuous}), (\cite{fowler2006continuous}) (\cite{fowler2012continuous}).

However, some of the newer methods, like using the MEAN stack or NodeJS are not yet fully covered.
The Question i will research in this thesis is, how can a modern webapplication automatized and deployed to continously integrate features and
changes. It is my goal to show how to establish a modern development operation cycle for webapplications using Node.js and the MEAN stack.
I will citeing the standard literature and methods for continuous integration and deployment and show how to build them in a safe and stable
way with Node.js and other Parts of the MEAN stack.
For this i will study the literatur that is available for continuous integration and autmatization to build a similar system with NodeJs.
To do so i will describe how to automatize builds and tests and how to use containerization to simplify the process. Based on the
implementation cycle of the neolexon webapplication, i will provide data to prove that this setup is capable of continous integration and deployment
ant that it runs in a stable and reliable way.

The start of the thesis will be a literature research and review about CI and DevOps in general, to build up the first two sections of this
thesis. The research will start on th 01. February and the first to sections will be ready on 21. February. In the following period
of three weeks i will set up the necessary backend for the CI and development process. I will describe the setup of the backend, tests and
CI methods in the next two sections. The third and forth section will be ready on the 10. of April. After this the last three weeks are
planed for the code examples and to provide data and comparisons from the tests to this thesis. According to this timetable the thesis will
be ready for submission on the first of mai.

Literature according to the first research (best reviewed from the references list at the end of the thesis):

\cite{meyer2014continuous}
\cite{schaefer2013continuous}
\cite{humble2010continuous}
\cite{fowler2006continuous}
\cite{fowler2012continuous}
\cite{duvall2007continuous}
\cite{stolberg2009enabling}
\cite{humble2010continuous}
\cite{staahl2014modeling}
\cite{maurer2002extreme}
\cite{hansen2015continuous}
\cite{pasquali2015deploying}
\cite{turnbull2014docker}
\cite{raj2015learning}
\cite{astels2003test}
\cite{beck2003test}
\cite{maximilien2003assessing}
\cite{janzen2005test}

\paragraph{Keywords:}
\textit{Continous Integration, Continuous Deployment, Automatization, Web Applications, Agile Development, XP, Extrem Programming, Docker, Containarization, DevOps }
