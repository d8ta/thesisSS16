\section{Exposé}

\subsection{Continuous Integration and Automatization in Modern Web Applications unsing NodeJS}


\subsection{Concept}
- 2–3 Seiten langer Fließtext
- Forschungsstand
- Forschungsfrage
- Ziele
- Vorgehen (Aufbau)
- Methode(n) 
- eigene Hypothesen

\subsubsection{Achievement}
The Mayor goal of this thesis is to show how to establish a modern development operation cycle for
web applications using Node.js and the MEAN stack. For this i will citeing the standard literature and methods for continuous
integration and deployment and show how to build them in a safe and stable way with Node.js and the
the other Parts of the MEAN stack.

\subsubsection{Research Question}
How to implement Continuous Integration, Continuous Development, Automatization and Testing for a Modern Web Applications
using the MEAN Stack and NodeJS in a stable way and practical way.

\subsubsection{Methods}
First i will describe the implementation process of continuous integration, based on research of the basic litratur for this
topic. Appart from this the thesis will provide data from the impementation cycle at the neolexon webapplication.
This data will show if the choosen methods and implementation are stable and if they will provide process to the deployment
and development of the application.

\subsection{Structure}
\subsubsection{Introduction to Agile Development and the Role of Development Operations in Modern Web Applications}
\subsubsection{The Foundations of Continuous Integration and Automatization in Development Operations}
\subsubsection{Development and Deployment with MEAN}
\subsubsection{Test Driven Development with MEAN}
\subsubsection{Containarization and Deployment with Docker}
\subsubsection{A Best Practice Approach for Building Continuous Integrations and Deployment with NodeJS}
\subsubsection{Appendix: Code example}

\subsection{Timetable}

\subsection{References}

Astels, Dave. 2003. Test driven development: A practical guide. Prentice Hall Professional Technical Reference.

Beck, Kent. 2003. Test-driven development: by example. Addison-Wesley Professional.

Duvall, Paul M, Steve Matyas, and Andrew Glover. 2007. Continuous integration: improving software quality and reducing risk. Pearson Education.

Fowler, Martin, and Matthew Foemmel. 2006. “Continuous integration.” Thought-Works) http://www.thoughtworks. com/Continuous Integration. pdf.

Hansen, Emil Alnæs Joakim Klevmo, and Henrik Heide. 2015. “Continuous Delivery with Docker.” Humble, Jez, and David Farley. 2010. Continuous delivery: reliable software releases through build, test, and deployment automation. Pearson Education.

Janzen, David, and Hossein Saiedian. 2005. “Test-driven development: Concepts, taxonomy, and future direction.” Computer, no. 9: 43–50.

Maurer, Frank, and Sebastien Martel. 2002. “Extreme programming: Rapid development for Webbased applications.” IEEE Internet computing, no. 1: 86–90.

Maximilien, E Michael, and Laurie Williams. 2003. “Assessing test-driven development at IBM.” In Software Engineering, 2003. Proceedings. 25th International Conference on, 564–569. IEEE.

Meyer, Michael. 2014. “Continuous integration and its tools.” Software, IEEE 31 (3): 14–16.

Pasquali, Sandro. 2015. Deploying Node. js. Packt Publishing Ltd.

Raj, Pethuru, Jeeva S Chelladhurai, and Vinod Singh. 2015. Learning Docker. Packt Publishing Ltd.

Schaefer, Andreas, Marc Reichenbach, and Dietmar Fey. 2013. “Continuous integration and au- tomation for DevOps.” In IAENG Transactions on Engineering Technologies, 345–358. Springer.

Stahl, Daniel, and Jan Bosch. 2014. “Modeling continuous integration practice differences in in- dustry software development.” Journal of Systems and Software 87:48–59.

Stolberg, Sean. 2009. “Enabling agile testing through continuous integration.” In Agile Conference, 2009. AGILE’09. 369–374. IEEE.

Turnbull, James. 2014. The Docker Book: Containerization is the new virtualization. James Turn- bull.

\newpage
