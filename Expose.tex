\section{Exposé}

\subsection{Continuous Integration and Automatization in Modern Web Applications unsing NodeJS and the MEAN Stack}


\subsection{Concept}
- 2–3 Seiten langer Fließtext

Today the need for stable webapplications is vital for companys and startups. Established businesses needs to be
available for their customers and users all the time. For Startups the availability of their service can be the difference
between failure and success. Webapplications and services needs to change and scale over time. Sometimes it is the need
to open up your service for more customers, like Runtastic needed to do, a small startup from Linz in austria who developed
a fitness application whose userbase was growing in a short periode of time. Somtimes the customerbase demands a 24/7
availability, like for the sales platform Amazon. Downtime is costly. In the best case it only costs money, in the worst case
it can be the failure of your business or startup. All the mentioned services and events have one thing in common. To provide this kind of zero downtime and continuous change in
your webapplication you need a reliable and stable automatisation for your services. Development operations are right now one of
the hot topics in webdevelopment. If used properly DevOps can provide this kind of zero downtime and progress in an application
\cite{humble2010continuous} \cite{duvall2007continuous}. To do so DevOps can use a variaty of techniques and methods to continuously
integrate, automatic test and deploy an applications \cite{meyer2014continuous}.
The Status Quo in Continuous Integration and development is based on the method of agile software development and extrem programming
(\textit{XP}) \cite{lindstrom2004extreme}. There is a reliable base of knowledge and literature for Continous Integration
()\cite{schaefer2013continuous}, \cite{fowler2006continuous} \cite{fowler2012continuous}). However, some of the newer methods, for example
MEAN stack or NodeJS are not yet fully covered.
The Question i will research in this thesis is, how can a modern webapplication be automatized and deployed to continously integrate features and
changes. It is my goal to show how to establish a modern development operation cycle for webapplications using Node.js and the MEAN stack.
I will citeing the standard literature and methods for continuous integration and deployment and show how to build them in a safe and stable
way with Node.js and other Parts of the MEAN stack.
For this i will study the literatur that is available for continuous integration and autmatization to build a similar system with NodeJs.
To do so i will describe how to autoatize builds and tests and how to use containerization to simplify the process. Data based on the
implementation cycle of the neolexon webapplication will provide data to prove that NodeJS is capable for continous integration and deployment
ant that this system is stable and reliable.

\subsection{Basic Structure}
\subsubsection{Introduction to Agile Development and the Role of Development Operations in Modern Web Applications}
\subsubsection{The Foundations of Continuous Integration and Automatization in Development Operations}
\subsubsection{Development and Deployment with MEAN}
\subsubsection{Test Driven Development with MEAN}
\subsubsection{Containerization and Deployment with Docker}
\subsubsection{Building Continuous Integrations and Deployment with NodeJS}
\subsubsection{Appendix: Code example}

\subsection{Timetable}
01.02.16 - 21.02.16 Literatur review and research about CI and DevOps; Starting the first two chapters of the Thesis
22.02.16 - 14.03.16 Setting up the Backend with tests and automatization; Describing the process in chapter 3 and 4
15.03.16 - 10.04.16 CI Cycle and Deployment and writing related chapters
11.04.16 - 02.05.16 Providing testdata and code examples of the practical part, writing the last chapter

\subsection{References}

Astels, Dave. 2003. Test driven development: A practical guide. Prentice Hall Professional Technical Reference.

Beck, Kent. 2003. Test-driven development: by example. Addison-Wesley Professional.

Duvall, Paul M, Steve Matyas, and Andrew Glover. 2007. Continuous integration: improving software quality and reducing risk. Pearson Education.

Fowler, Martin, and Matthew Foemmel. 2006. “Continuous integration.” Thought-Works) http://www.thoughtworks. com/Continuous Integration. pdf.

Hansen, Emil Alnæs Joakim Klevmo, and Henrik Heide. 2015. “Continuous Delivery with Docker.” Humble, Jez, and David Farley. 2010. Continuous delivery: reliable software releases through build, test, and deployment automation. Pearson Education.

Janzen, David, and Hossein Saiedian. 2005. “Test-driven development: Concepts, taxonomy, and future direction.” Computer, no. 9: 43–50.

Maurer, Frank, and Sebastien Martel. 2002. “Extreme programming: Rapid development for Webbased applications.” IEEE Internet computing, no. 1: 86–90.

Maximilien, E Michael, and Laurie Williams. 2003. “Assessing test-driven development at IBM.” In Software Engineering, 2003. Proceedings. 25th International Conference on, 564–569. IEEE.

Meyer, Michael. 2014. “Continuous integration and its tools.” Software, IEEE 31 (3): 14–16.

Pasquali, Sandro. 2015. Deploying Node. js. Packt Publishing Ltd.

Raj, Pethuru, Jeeva S Chelladhurai, and Vinod Singh. 2015. Learning Docker. Packt Publishing Ltd.

Schaefer, Andreas, Marc Reichenbach, and Dietmar Fey. 2013. “Continuous integration and au- tomation for DevOps.” In IAENG Transactions on Engineering Technologies, 345–358. Springer.

Stahl, Daniel, and Jan Bosch. 2014. “Modeling continuous integration practice differences in in- dustry software development.” Journal of Systems and Software 87:48–59.

Stolberg, Sean. 2009. “Enabling agile testing through continuous integration.” In Agile Conference, 2009. AGILE’09. 369–374. IEEE.

Turnbull, James. 2014. The Docker Book: Containerization is the new virtualization. James Turn- bull.

\newpage
