\selectlanguage{ngerman}
\section*{Kurzfassung}
\begin{tabular}{l l}
Vor- und Zuname:& Daniel RAUDSCHUS\\
Institution: & FH Salzburg\\
Studiengang: &  Bachelor MultiMediaTechnology\\
Titel der Bachelorarbeit: & Continuous Integration and Automatization in Modern Web Applications\\
Begutachter: & Hannes Moser\\
\end{tabular}
\vspace{0.5cm}

Eine der wichtigsten Eigenschaften moderner Web Applikationen ist Skallierbarkeit und Zero Downtime Managment. Da die Standardtechniken anderer Ingenieurzweige
einen deutlichen Unterschied zu den Abläufen von Softwareprojekten aufweisen, sollten in der Softwareentwicklung andere Methoden angewandt werden. Einer der
Hauptunterschiede zu anderen Ingeneurzweigen ist die Trennung von Design und Konstruktion. Um neue Methoden für die Softwareentwicklung nutzen zu können
wurde die agile Softwareentwicklung geschaffen. Neben der Berücksichtigung des grösseren Designprozesses, ist die nicht-trennbarkeit von Design und Konstruktion
einer der Hauptargumente agiler Methodiken. Diese ermöglichen Flexibilät, was durch die ständigen Veränderungen, denen Software unterworfen ist eine grosse
Rolle spielt. Ein weiterer Punkt agiler Softwareentwicklung ist die kontinuierliche Integration neuen Codes in bestehende Projekte. Diese Arbeit
beschreibt einen möglichen Ansatz um Continuous Integration Systeme in MEANJS und Node JS Projekte zu integrieren.

\paragraph{Schlagwörter:}
Continuous Integration, Continuous Deployment, Automatization, Web Applications, Agile Development, MEAN, MongoDB, Express, AngularJS, NodeJS, DevOps, Software Engineering, Web Development
\selectlanguage{english}
