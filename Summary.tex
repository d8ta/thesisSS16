\selectlanguage{ngerman}
\section*{Kurzfassung}
\begin{tabular}{l l}
Vor- und Zuname:& Daniel RAUDSCHUS\\
Institution: & FH Salzburg\\
Studiengang: &  Bachelor MultiMediaTechnology\\
Titel der Bachelorarbeit: & Continuous Integration with NodeJS\\
Begutachter: & Hannes Moser\\
\end{tabular}
\vspace{0.5cm}

Eine der wichtigsten Eigenschaften moderner Web Applikationen ist Skallierbarkeit und Zero Downtime Managment. Da die Standardtechniken andere Ingenieurszweige
einen deutlichen Unterschied zu den Abläufen von Sofware Projekten aufweisen, sollten in der Softwareentwicklung andere Methoden angewand werden. Einer der
Hauptunterschiede zu anderen Ingeneurzweigen ist die Trennung von Design und Konstruktion. Um neue Methoden für die Softwareentwicklung nutzen zu können
wurde die agile Softwareentwicklung geschaffen. Neben der Berücksichtigung des grossen Designprozesses und der nicht-trennbarkeit von Design und Konstruktion
inst einer der der Hauptargumente für diese Technik, die enthaltende Flexibilät. Software ist grossen Veränderungen unterworfen, schon im Design und
Konstruktionsprozess. Ein wichtiger Punkt agiler Softwareentwicklung ist die Continuierliche Integration neuen Codes in bestehende Projekte. Diese Arbeit
beschreibt einen best practise Ansatz um Continuous Integration Systeme in MEANJS und Node JS Projekte zu integrieren.

\paragraph{Schlagwörter:}
Continuous Integration, Continuous Deployment, Automatization, Web Applications, Agile Development, MEAN, MongoDB, Express, AngularJS, NodeJS, DevOps, Software Engineering, Web Development
\selectlanguage{english}
